\noindent Nowadays, many websites and applications are facing the important issue of efficient synchronization of text editors \citep{MOSCICKI20181052}. Ideally, this kind of cloud storage can provide users with a convenient environment. In file sharing scenario, when users are using documents store-oriented programs and websites, they will inevitably encounter some conflicts and contradictions. Researchers have made many attempts and breakthroughs in recent years, and many successful programs and applications have emerged in this field, such as Dropbox, Google Drive, Microsoft Drive, etc. In addition, Dropbox announced that it has hit a new milestone of 300 million users, adding 100 million users in just six months \citep{7417235}.


\vspace{0.3cm}
\noindent Although the number of users of dropbox is very large and considerable, objectively speaking, Google Drive seems more comprehensive and have more functions than Dropbox. Specifically, Google Drive has sophisticated and powerful sharing capabilities, and enables online editing and support for documents in different formats \citep{googledrive}.

\begin{comment}
Jakub T. Mo艣cicki, Luca Mascetti,
Cloud storage services for file synchronization and sharing in science, education and research,
Future Generation Computer Systems,
Volume 78, Part 3,
2018,
Pages 1052-1054,
ISSN 0167-739X,
https://doi.org/10.1016/j.future.2017.09.019.
\end{comment}

\vspace{0.3cm}
\noindent With the collaboration feature, no matter where the participants are, whether the user is using the mobile phone or the web page, he or she can write a document together with teammates, just like sitting together to discuss. Users can see other all editorial behaviors of everyone, and can be commented or discussed in real time, this is an efficient way of working, and also the main function our group project want to achieve. The core of many studies like \citep{parker2001method} complements the core content of our project, which provides different reliable methods and algorithms to implement synchronization logic.

\vspace{0.3cm}
\noindent In a previous research \citep{1498448}, a single-round protocol called file synchronization was mentioned. This protocol provides many major improvements and has proven to be effective. But this method is more based on the Linux platform. However, in our project, we need to find relevant experiences and algorithms to complete an iOS-based client to assist users to solve synchronization problems.

\vspace{0.3cm}
\noindent Mobile devices have changed how we conceive software. There is a great range of development alternatives. At present, mobile web applications have been widely used, and there are many examples and valuable experiences in the current development background and field \citep{7128878}.
 
\vspace{0.3cm}
\noindent There are many different options for the structure and framework of such synchronization systems. Many programming languages and architectures have ready-made open source libraries for developers to use. In addition, some documents like \citep{hao2017} can provide good references and ideas. Although the specific purpose and function are different, this remote device virtual maintenance training system is a complete system based on ThinkPHP and MVC framework. It details how the ThinkPHP and mvc frameworks are combined and put into use.

\vspace{0.3cm}
\noindent In the future, applications and achievements of this kind of synchronization function will increase rapidly with more comprehensive improvements of their features. How to achieve synchronous editing logic design will be more and more transparent. 


%Current work‘s status and comparison in this field(done)

%ways and functions to achieve sync logic(main part)

%relative algorithms

%advantages and applications(done)

%drawbacks and limitations of other researches, & future work
