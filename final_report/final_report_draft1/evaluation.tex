\subsection{Advantages}
\begin{itemize}
    \item After our use and testing, we found and confirmed that mvc development is very suitable for our project. In the process of development, the code under the mvc framework is neat and simple, easy to modify, saving a lot of time and manpower.
    \item Our code specification, the proportion of comments is appropriate.
    \item In addition to special circumstances such as temporary scheduling, our team regularly organizes group meetings on Tuesdays and Fridays to discuss progress and learn and solve current problems. The members of the group are well-coordinated, and everyone has contributed to this group project and made their own contributions without disputes and irresponsible members.
    \item Before doing this project, the members of the group did not contact or use github. In the past few months, we gradually learned how to use github to save our own code and files from the blank state at the beginning.
    
\end{itemize}


\subsection{Disadvantages}
\begin{itemize}
    \item In the editing function of the file, at present we can only edit the .txt file, and the files in other formats have not been implemented yet.
    \item Although we have been learning how to use github, our technology is not very skilled compared to the students who already have experience and experience. The frequency of using github is not particularly high. Since we have more meetings per week, we also have other software to communicate within the group, so there is no balance between github interaction and other ways of communication.
\end{itemize}

\subsection{Differences from the Initial Idea}
\begin{itemize}
    \item Before we start doing this project, we intend to use the OT algorithm, which is a technology for supporting collaborative computing functions and applications to achieve collaboration 
    \item Later, in the process of studying this algorithm, we found that although this algorithm has advantages in operation conversion and processing collaborative editing, for our project, this algorithm is too complicated, and we want to achieve the core functions. It is not completely compatible. If we want to achieve simultaneous editing of files, the focus is not on OT-oriented design. That is, in design, the principle that is more conducive to the implementation of OT algorithm is the first, and the principle of reducing network transmission is second. In practice, it is necessary to have enough storage space to buffer the traffic difference between the two streams. The client OT must be a message that is bypassed, not a message processed by the database. This is more difficult for our system design and gpu requirements. Firstly, we searched for relevant materials about OT technology from the Internet since our team members are not familiar with this technology. There are various online instructions about OT algorithms using different computer languages such as Java, PHP, and Python.  But unfortunately, the tutorials using swift that our system gonna use are rare. In the process of trying to use OT, although it is easy to understand the principle of this technology, the actual operation is quite complicated. The reason why is that the OT algorithm is dependent on the order, and the different order will lead to different results. In addition, there will be some problem if the original file changed before the modification submit as the modification is based on the old version. Therefore, OT technology did not work well in our system, the bugs always appeared and we even cannot find solutions to solve them.So after finding a more simple and easy to implement method, we gave up the OT algorithm.
    
    \item We originally planned to use java to write mobile apps. But then we chose iOS to develop more commonly used swift. Swift supports both object-oriented programming and functional programming. Swift is more powerful than Java and more user-friendly.In addition, xcode has a ready-made swift framework that makes our use and design easier.
\end{itemize}



\subsection{Future Work}

The future work contains four sides:

\begin{itemize}
    \item The function downloading is needed when we only have uploading and sharing currently. To implement this function, there are two typical ways. Firstly, we think about the standard URL download which is embedding an URL hyperlink in the web page and then downloading using a standard HTTP GET request. However, the drawback of this method is that the path of the file is completely exposed, which leads to a low security and privacy of the website. The second method is to submit the form, submitting the parameters to the server-side dynamic script by using POST request, and then the server-side script returns the output binary stream to the browser for download. This downloading method is not only available for the specific files on server but also for the data that dynamically generated by server. We will experiment with both methods and choose the one that works best.
    %\item Various privilege will be set. Now, the users can directly view and modify the file when the file creator add them. In the future, we will divide permissions into read, write, and operation which contains deleting, downloading and sharing file. The owner of a file will have both these three permissions on the file initially. Besides, only the owner can edit the privilege of the file and he/she can let other users to only read, only operate or do nothing on the specific file as they want. To achieve this function, we will need to build the privilege table for each file, and the table should be automatically generated and deleted with the file.
    \item At present our project can only support the operation on the file that browser can open directly like txt. We will improve the system to handle different types of files like excel and pdf. 
    \item Our project is based on local database without server to save time and money as the lease of the server is expensive, and the connection to the server is complicated, which involves licensing, installation, maintenance, support, and patching associated with the operating system. However, Serverless computing is not suitable for workloads with high computing performance requirements due to the limitations on resources, and there are many application components in the serverless architecture, so the system is also in high risk to be attacked. Thus, in the future work, we will try to connect our system to the server.

\end{itemize}

\noindent For the web front-end design, we use the GIF image as the background of login page with a cute logo in order to make the webpage more vibrant, but the background is not coherent as it is hard to find a high-definition picture with coherent animation. when entering the index page, we originally prepared to use the uniform background color, but this is easy for users to feel boring, so we set different background images for each function page, and the pictures are all about nature, which makes the webpage more ornamental and eye-catching. For the choice of detail colors, we use the dark blue and gray that are generally accepted by the public.





