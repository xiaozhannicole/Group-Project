\noindent To facilitate our group work,we keep finding proper ways to solve potential difficulties and made specific plan to have meeting every week. In this section, we suppose to describe our teamwork in two aspects.

\subsection{Tools}
\noindent When it comes to what tools we choose to communicate with team members. We find face-to-face discussions is the most efficient way, followed by using Wechat and Git Hub.
\begin{itemize}
    \item Face-to-Face discussion: which are regarded by all the teammates as the most efficient way to solve problems and learn some knowledge together.
    \item Wechat: mainly used to share learning materials and make decision about meeting time and place. In addition, Wechat is a good tool we used to supplement the face-to-face communication because we could not spend all the time together.
    \item GitHub for version control: used to store main codes and report files and coordinate together with source code and commits.
\end{itemize}
\vspace{0.3cm}

\subsection{Main Processes}

\noindent First of all, our group holds one or two group discussions every week at the Waterloo campus library after we discuss the time that suitable for everyone in WeChat. Each meeting lasts about three hours.During the meeting time, we firstly make a simple report on the work done by everyone from the last meeting to the present, and then evaluate the completion rate of each person and improve the task allocation of each meeting, so that the work can be assigned according to each person's ability and the individual work will not be too much or too little. Then we will collect the unfinished tasks and determine the new tasks and goals we need to complete before the next meeting. We usually assign tasks through self-recommendation after discussing about all the upgrades.
 
 
 %However, there are some occasions that no one is interested in or good at these tasks. At this time, there is usually a teammate serves as an interim team leader to assign tasks. At the beginning, our face-to-face discussion was extremely inefficient. In two-hour meeting, because of mutual unfamiliarity and lack of clear purpose, everyone was gossiping and enjoying their self-study. After about three to four meetings, everyone became more active since we were familiar with each other. We were willing to speak out our own ideas even if we were not sure whether they were correct and we were willing to take the initiative to undertake tasks. When one of the member encounter difficulties, others will be enthusiastic to help him to find relevant materials and solutions. Our face-to-face discussions have therefore become more efficient and meaningful.
\vspace{0.3cm}


\noindent We created a conversation group in the WeChat application which is very helpful and efficient. Face-to-face meetings are not enough and we need dialogue anytime and anywhere. Everyone's work is interrelated, such as the front-end design and back-end response of the website, so the team members should communicate and negotiate frequently. In the group we will discuss the troubles we encountered when completing the assigned tasks after the meeting. At this time, teammates who are good at this field or experienced will provide solutions or suggestions. Thus, we don't have to wait until the next meeting but can quickly resolve problems without affecting the follow-up work. In addition, we will also share relevant learning materials and insights in the WeChat group so that others can gain knowledge and make progress together.
On the use of GitHub, every member in our group have exposed to this tool, but only for simple upload and download operations. Therefore, in the initial establishment of the group, everyone created a new group, which was definitely a wrong operation. Later, we got familiar with the fork and pull request functions after reading the online information and watching the explaining video that recommended by lecturer. These two operations are very effective for our group work, By forking everyone got an individual project based on group project, which is essentially a new branch of the original project. we made improvements in our own personal projects, but they do not affect the code and structure of the group project. Then made a pull request based on Fork to the 'Captain', everyone in the group can see the changes directly and then comment on them. After everyone feels satisfied, the team leader will accept the improvements and merge them into the original project.
\vspace{0.3cm}

\noindent In the previous month, we discussed and recommended each person's work mainly through face-to-face meeting, so as you can see, our group interactions in GitHub were only pull request and directly merger in this period. However, the lecturer requested a more positive comment on each modification of the team projects, and we started using the comments function gradually.
\vspace{0.3cm}

\noindent In general, the coordination between each team members getting better and stronger when everyone was concern about their individual part at the beginning. Besides, the use of relevant tools are also more and more skilled, each member has a more profound understanding of aggregation and the process of teamwork.
